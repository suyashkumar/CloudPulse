\documentclass[margin=25pt]{standalone}
\usepackage{amsmath}    % load AMS-Math package
\usepackage[]{circuitikz}
\tikzstyle{every path}=[line width=0.5pt,line cap=round,line join=round]
\ctikzset{bipoles/thickness=1}
\ctikzset{bipoles/length=0.8cm}



% Define Custom Commands
\newcommand{\tab}[1]{\hspace{.07\textwidth}\rlap{#1}} % Tab
\newcommand{\tabm}[1]{\hspace{.05\textwidth}\rlap{#1}} 
\newcommand{\tsub}[1]{\textsubscript{{#1}}}           % Shorter subscript command
\newcommand{\e}[1]{\ensuremath{\times 10^{#1}}}
\newcommand{\inclfig}[2]{\begin{figure}[h]\begin{center}\includegraphics[scale=#2]{#1}\end{center}\end{figure}}
\newcommand{\inclfigc}[3]{\begin{figure}[h]\begin{center}\includegraphics[scale=#2]{#1} \caption{#3} \end{center}\end{figure}}
\newcommand{\inclfigcl}[4]{\begin{figure}[h]\begin{center}\includegraphics[scale=#2]{#1} \caption{#3} \label{fig:#4} \end{center}\end{figure}}
\newcommand{\unit}[1]{\ensuremath{\,\mathrm{#1}}}
\newcommand{\kOhm}[0]{\ensuremath{\, \mathrm{k}\Omega}}
\newcommand{\eq}[0]{\ensuremath{=}}
\newcommand{\mV}[0]{\ensuremath{\,\mathrm{mV}}}
\newcommand{\myopampsize}[2] % #1 = name , #2 = scaling factor
{\draw[thick] (#1){};
 \filldraw[white] (#1.+) +(12pt,2pt) circle(3pt)  
                  (#1.-) +(12pt,-2pt) circle(3pt);
 \draw[]          (#1.+) +(12pt,0pt) node(){\scalebox{#2}{$\mathbf{+}$}}
                  (#1.-) +(12pt,0pt) node(){\scalebox{#2}{$\mathbf{-}$}};
}
\newcommand{\drawOA}[3]{
  \draw[]
  (#1,#2) node[op amp,xscale=0.7,yscale=0.7] (#3){};
  \myopampsize{#3}{0.9}

}
\newcommand{\drawROA}[3]{
  \draw[]
  (#1,#2) node[op amp,xscale=0.7,yscale=-0.7] (#3){};
  \myopampsize{#3}{0.9}

}

\begin{document}


\begin{circuitikz}
%\draw[]
%(3,-0.5) node[op amp,xscale=0.7,yscale=0.7] (opamp1){};
\drawOA{2}{-4.345}{opamp1}
\drawROA{8}{-4.688}{opamp2}
\drawROA{14}{-5.031}{opamp3}
\draw 
(0,0.4)node[]{+5V};
\draw[]
(0,0) node[currarrow, rotate=90]{} to [R,l=$R_1\eq590\Omega$] ++(0,-2) to
[leDo,l=IR LED] ++(0,-1) to ++(0,-1) node[circ] (join) {} to ++(0,-1) node[ground] {}
(join) to [pDo,l=In] ++(1,0) to (opamp1.-)
(opamp1.-) to ++(0,1) to [R,l=$\,\,\,\,\,\,\,\,\,\,\,\,\,\,\,\,\,\,\,\,\,R_2\eq1.5\mathrm{M}\Omega$] ++(1.5,0) to ++(0,-1.34)
(opamp1.out) to [C,l=$\,\,C_1\eq0.1\mu F$] ++(1.5,0)
to ++(1,0) node[circ] (join2) {} to [R,l=\footnotesize	$R_3\eq3.3\mathrm{M}\Omega$] ++(0,-1) to
++(0,-0.2) node[ground]{}
(join2) to (opamp2.+)
(opamp2.out) to [R,l=\scriptsize$R_4\eq1.5\mathrm{M}\Omega$] ++(0,-1) to ++(0,0) node[circ] (join3) {}
to [R,l=$R_5\eq100k\Omega$] ++(0,-1) node[ground]{}
(join3) to ++(-1.668,0) to ++(0,0.651)
(opamp2.out) to ++(0.5,0) to [R,l=\small$R_6\eq150\kOhm$] ++(1,0) to ++(0.7,0) node[circ] (join4) {}
(join4) to [C,l=\small$C_1\eq 0.1 \mu F$] ++(0,-0.8) to ++(0,-0.005) node[ground]{} 
(join4) to (opamp3.+)
(opamp3.-) to ++(0,-0.8) to ++(1.7,0) to ++(0,1.14);
\draw
(opamp1.+) to ++(0,-0.5) node[ground]{};
\draw
(opamp3.out) to ++(.8,0) node[ocirc]{$\,\,\,V_{out}\,\,\,$};
\end{circuitikz}


\end{document}
